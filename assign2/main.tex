\let\negmedspace\undefined
\let\negthickspace\undefined
\documentclass[journal,12pt,onecolumn]{IEEEtran}
\usepackage{cite}
\usepackage{amsmath,amssymb,amsfonts,amsthm}
\usepackage{algorithmic}
\usepackage{graphicx}
\usepackage{textcomp}
\usepackage{xcolor}
\usepackage{txfonts}
\usepackage{listings}
\usepackage{enumitem}
\usepackage{mathtools}
\usepackage{gensymb}
\usepackage[breaklinks=true]{hyperref}
\usepackage{tkz-euclide} % loads  TikZ and tkz-base
\usepackage{listings}



\newtheorem{theorem}{Theorem}[section]
\newtheorem{problem}{Problem}
\newtheorem{proposition}{Proposition}[section]
\newtheorem{lemma}{Lemma}[section]
\newtheorem{corollary}[theorem]{Corollary}
\newtheorem{example}{Example}[section]
\newtheorem{definition}[problem]{Definition}
%\newtheorem{thm}{Theorem}[section] 
%\newtheorem{defn}[thm]{Definition}
%\newtheorem{algorithm}{Algorithm}[section]
%\newtheorem{cor}{Corollary}
\newcommand{\BEQA}{\begin{eqnarray}}
\newcommand{\EEQA}{\end{eqnarray}}
\newcommand{\define}{\stackrel{\triangle}{=}}
\theoremstyle{remark}
\newtheorem{rem}{Remark}
%\bibliographystyle{ieeetr}
\begin{document}
%
\providecommand{\pr}[1]{\ensuremath{\Pr\left(#1\right)}}
\providecommand{\prt}[2]{\ensuremath{p_{#1}^{\left(#2\right)} }}        % own macro for this question
\providecommand{\qfunc}[1]{\ensuremath{Q\left(#1\right)}}
\providecommand{\sbrak}[1]{\ensuremath{{}\left[#1\right]}}
\providecommand{\lsbrak}[1]{\ensuremath{{}\left[#1\right.}}
\providecommand{\rsbrak}[1]{\ensuremath{{}\left.#1\right]}}
\providecommand{\brak}[1]{\ensuremath{\left(#1\right)}}
\providecommand{\lbrak}[1]{\ensuremath{\left(#1\right.}}
\providecommand{\rbrak}[1]{\ensuremath{\left.#1\right)}}
\providecommand{\cbrak}[1]{\ensuremath{\left\{#1\right\}}}
\providecommand{\lcbrak}[1]{\ensuremath{\left\{#1\right.}}
\providecommand{\rcbrak}[1]{\ensuremath{\left.#1\right\}}}
\newcommand{\sgn}{\mathop{\mathrm{sgn}}}
\providecommand{\abs}[1]{\left\vert#1\right\vert}
\providecommand{\res}[1]{\Res\displaylimits_{#1}} 
\providecommand{\norm}[1]{\left\lVert#1\right\rVert}
%\providecommand{\norm}[1]{\lVert#1\rVert}
\providecommand{\mtx}[1]{\mathbf{#1}}
\providecommand{\mean}[1]{E\left[ #1 \right]}
\providecommand{\cond}[2]{#1\middle|#2}
\providecommand{\fourier}{\overset{\mathcal{F}}{ \rightleftharpoons}}
\newenvironment{amatrix}[1]{%
  \left(\begin{array}{@{}*{#1}{c}|c@{}}
}{%
  \end{array}\right)
}
%\providecommand{\hilbert}{\overset{\mathcal{H}}{ \rightleftharpoons}}
%\providecommand{\system}{\overset{\mathcal{H}}{ \longleftrightarrow}}
	%\newcommand{\solution}[2]{\textbf{Solution:}{#1}}
\newcommand{\solution}{\noindent \textbf{Solution: }}
\newcommand{\cosec}{\,\text{cosec}\,}
\providecommand{\dec}[2]{\ensuremath{\overset{#1}{\underset{#2}{\gtrless}}}}
\newcommand{\myvec}[1]{\ensuremath{\begin{pmatrix}#1\end{pmatrix}}}
\newcommand{\mydet}[1]{\ensuremath{\begin{vmatrix}#1\end{vmatrix}}}
\newcommand{\myaugvec}[2]{\ensuremath{\begin{amatrix}{#1}#2\end{amatrix}}}
\providecommand{\rank}{\text{rank}}
\providecommand{\pr}[1]{\ensuremath{\Pr\left(#1\right)}}
\providecommand{\qfunc}[1]{\ensuremath{Q\left(#1\right)}}
	\newcommand*{\permcomb}[4][0mu]{{{}^{#3}\mkern#1#2_{#4}}}
\newcommand*{\perm}[1][-3mu]{\permcomb[#1]{P}}
\newcommand*{\comb}[1][-1mu]{\permcomb[#1]{C}}
\providecommand{\qfunc}[1]{\ensuremath{Q\left(#1\right)}}
\providecommand{\gauss}[2]{\mathcal{N}\ensuremath{\left(#1,#2\right)}}
\providecommand{\diff}[2]{\ensuremath{\frac{d{#1}}{d{#2}}}}
\providecommand{\myceil}[1]{\left \lceil #1 \right \rceil }
\newcommand\figref{Fig.~\ref}
\newcommand\tabref{Table~\ref}
\newcommand{\sinc}{\,\text{sinc}\,}
\newcommand{\rect}{\,\text{rect}\,}
%%
%	%\newcommand{\solution}[2]{\textbf{Solution:}{#1}}
%\newcommand{\solution}{\noindent \textbf{Solution: }}
%\newcommand{\cosec}{\,\text{cosec}\,}
%\numberwithin{equation}{section}
%\numberwithin{equation}{subsection}
%\numberwithin{problem}{section}
%\numberwithin{definition}{section}
%\makeatletter
%\@addtoreset{figure}{problem}
%\makeatother

%\let\StandardTheFigure\thefigure
\let\vec\mathbf

\bibliographystyle{IEEEtran}


\vspace{3cm}



\bigskip

\renewcommand{\thefigure}{\theenumi}
\renewcommand{\thetable}{\theenumi}
%\renewcommand{\theequation}{\theenumi}

Question:Suppose than 90\% of people are right-handed. What is the probability that atmost 6 of a random sample of 10 people are right-handed? 
\\ \solution:To calculate the probability of having at most 6 right-handed individuals out of a random sample of 10 people, given that the probability of an individual being right-handed is 90\% (0.9), we can use the binomial distribution formula:

\begin{equation}
P(X = k) = \binom{n}{k} \cdot p^k \cdot (1 - p)^{n - k}
\end{equation}

where:
\begin{align}
n & = \text{total number of trials (sample size)} = 10 \\
k & = \text{number of successes we're interested in (at most 6)} \\
p & = \text{probability of success (probability of a person being right-handed)} = 0.9 \\
\binom{n}{k} & = \text{binomial coefficient} = \frac{n!}{k! \cdot (n - k)!}
\end{align}

We need to calculate the probabilities for $ k = 0, 1, 2, 3, 4, 5 $ and $6$, and then sum them up to get the probability of getting at most 6 right-handed individuals.
Let's calculate it step by step:

\begin{align}
\text{For } k = 0: & \quad P(X = 0)=\binom{10}{0} \cdot (0.9)^0 \cdot (1 - 0.9)^{10 - 0} \\
                                   &=(0.1)^{10} \\
                                   &=0.0000000001\\                             
\text{For } k = 1: & \quad P(X = 1)= \binom{10}{1} \cdot (0.9)^1 \cdot (1 - 0.9)^{10 - 1} \\
&=10(0.9)(0.1)^{9} \\
&=9(0.1)^{9}       \\                            
\text{For } k = 2: & \quad P(X = 2)= \binom{10}{2} \cdot (0.9)^2 \cdot (1 - 0.9)^{10 - 2} \\
&=45(0.9)^{2}(0.1)^{8}\\
&=36.45(0.1)^{8}\\
\text{For } k = 3: & \quad P(X = 3)= \binom{10}{3} \cdot (0.9)^3 \cdot (1 - 0.9)^{10 - 3} \\
&=120(0.9)^{3}(0.1)^{7}\\
&=87.48(0.1)^{7}\\
\text{For } k = 4: & \quad P(X = 4)= \binom{10}{4} \cdot (0.9)^4 \cdot (1 - 0.9)^{10 - 4} \\
&=210(0.9)^{4}(0.1)^{6}\\
&=137.781(0.1)^{6}\\
\text{For } k = 5: & \quad P(X = 5)= \binom{10}{5} \cdot (0.9)^5 \cdot (1 - 0.9)^{10 - 5} \\
&=252(0.9)^{5}(0.1)^{5}\\
&=148.80348(0.1)^{5}\\
\text{For } k = 6: & \quad P(X = 6)= \binom{10}{6} \cdot (0.9)^6 \cdot (1 - 0.9)^{10 - 6}\\
&=210(0.9)^{6}(0.1)^{4}\\
&=111.60261(0.1)^{4}
\end{align}
Finally, the probability of getting at most 6 right-handed individuals is the sum of the probabilities for all these cases:
\begin{align}
P(\text{at most 6}) &= P(X = 0) + P(X = 1) + P(X = 2) + P(X = 3) + P(X = 4) + P(X = 5) + P(X = 6)\\
&=9(0.1)^{9} +36.45(0.1)^{8}+87.48(0.1)^{7} + 137.781(0.1)^{6}+148.80348(0.1)^{5}+111.60261(0.1)^{4}\\
&=0.000000009 + 0.0000003645+0.000008748+0.000137781+0.0014880348+0.011160261 \\
&=0.0127951973
\end{align}
\end{document}



